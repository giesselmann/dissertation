\chapter{Discussion}
\label{cha:summary}

But could you not also do that with short reads?
Leads to research question:
Would long reads not be the more appropriate technology or useful for confirmation?

Research question: Usability of nanopore sequencing, what is working, limiting factors, unique novel applications

Answer: Nanopore adapted for genome assembly and widely used to sequence bacterial and viral genomes.
Genome size matters, bioinformatic reproducibility and quality / maintenance of developed tools
Compare to NGS tools which can be finished, nanopore tools need to follow fast paced field


STRique: Nanopore for inaccessible part of genome
integrate SV detection with epigenetic modifications
digital method of previously analogue workflow, improving accuracy and turnaround time
shortcomings of neural network basecalling algorithms
counting on signal level, signal manipulation to facilitate conventional methylation calling
HMM methylation detection on repeat
tested on FMR1 and c9orf72, but applies to any repeat in the genome.
Reference to large e.g. satellite repeats, where sequence based methods will be accurate.
limitations: runtime and dependencies, reported by users using PCR amplified input, less relevant for clinical diagnostic application, caused by signal level algorithm and python HMM package
HMM needs to support states with continuous emission distributions 
count depends on flanking signal detection, requiring sufficient coverage for filtering
Accuracy only validated <100 repeats, larger repeats match southern blotting

Nanopype: Processing to stay up to date with updating software
storage to handle raw data, x times larger than fastq from NGS
snakemake over shell scripts for modularity, supporting container, multiple cluster backends
Build and functionality tests are very rare!
Interchangeable tools for lab focused on development
weakness: trying to maintain increasing number of integrated tools and their dependencies, example pytorch/tensorflow v1/v2 for neural networks
Future work: freeze core pipeline, develop NanopypeXT which uses Nanopype as sub-workflow for very specific or complex to maintain tools

Literature: Throughput and accuracy increasingly suitable for application to large plant and mammalian genomes
Assembly and structural variant detection are established but will further benefit from accuracy and even longer reads
Limitations on wet-lab side with high molecular weight DNA and amount of input material
Limitations on bioinformatic, storage and upload, refer to STRique data being in Figshare repository
Base modification visibility long known, lack of robust and accurate tools
RNA modifications example of frequently mentioned potentials with only few applications up to date

Future work based on pipeline backend and baseline set with signal annotation
Base modification framework on event data without dependency to basecaller generation
Base caller based on novel transformer neural networks with impressive results in translation and speech to text







