\chapter{Discussion}
\label{cha:summary}

\textit{But could you not also do that with short reads?} - Expressing an initial scepticism against a novel technology, nanopore sequencing is, six years after the first release in 2014, still termed the emerging technology.
Next-generation sequencing remains the established and widely undisputed reference method.
However, after significant improvements, addressing shortcomings of throughput and accuracy, both third-generation platforms have evolved to become the primary technology for genome assembly and are commonly used for the analysis of viral and bacterial genomes.
Yet, a broader replacement of NGS methods is not in sight, raising the question, why the introductory question is not more frequently re-phrased to: \textit{Would nanopore here not be the more suitable platform?} Or: \textit{Would you not confirm this by using nanopore sequencing?}

The underlying research question of this work is therefore to analyze unique use-cases for nanopore sequencing, identify key limitations and provide a comprehensive investigation of the current status.
Despite many improvements, the application of nanopore sequencing in the context of large mammalian and plant genomes remains challenging.
At the same time are the benefits of long reads known: Reliable alignments in repetitive regions, no duplicates and biases by library amplification and the linkage of distant genetic and epigenetic features on single molecule level.
Yet, the stable throughput, higher read accuracy and established workflows of NGS technologies justify the commitment to nanopore sequencing only for applications infeasible with short reads.

An initial niche to gain traction at applying nanopore sequencing is the analysis of repetitive regions in the human genome.
For comparison, using short reads, the reachable part of a repeat is limited by the read length, leading to ambiguous alignments once a read contains only repeat sequence.
Short tandem repeats (STRs) are accumulations of three to six nucleotide long sequence patterns, in disease cases expanded to hundreds of copies.
The epigenetic analysis of repeats is a unique feature of nanopore sequencing, given that SMRT sequencing is on single-molecule level not sufficiently sensitive to 5-methylcytosine.
Repeat detection by sequencing is the digital advancement over southern blot based quantification, increasing the accuracy and reducing the turnaround time.
\textit{STRique} is a bioinformatic analysis software developed to integrate the quantification of short tandem repeats with their methylation state on individual read level.
The algorithmic key features of \textit{STRique} are raw signal alignment and sequence to signal annotation.



shortcomings of neural network basecalling algorithms
counting on signal level, signal manipulation to facilitate conventional methylation calling
HMM methylation detection on repeat
tested on FMR1 and c9orf72, but applies to any repeat in the genome.
Reference to large e.g. satellite repeats, where sequence based methods will be accurate.
limitations: runtime and dependencies, reported by users using PCR amplified input, less relevant for clinical diagnostic application, caused by signal level algorithm and python HMM package
HMM needs to support states with continuous emission distributions 
count depends on flanking signal detection, requiring sufficient coverage for filtering
Accuracy only validated <100 repeats, larger repeats match southern blotting

Nanopype: Processing to stay up to date with updating software
storage to handle raw data, x times larger than fastq from NGS
snakemake over shell scripts for modularity, supporting container, multiple cluster backends
Build and functionality tests are very rare!
Interchangeable tools for lab focused on development
weakness: trying to maintain increasing number of integrated tools and their dependencies, example pytorch/tensorflow v1/v2 for neural networks
Future work: freeze core pipeline, develop NanopypeXT which uses Nanopype as sub-workflow for very specific or complex to maintain tools

Literature: Throughput and accuracy increasingly suitable for application to large plant and mammalian genomes
Assembly and structural variant detection are established but will further benefit from accuracy and even longer reads
Limitations on wet-lab side with high molecular weight DNA and amount of input material
Limitations on bioinformatic, storage and upload, refer to STRique data being in Figshare repository
Base modification visibility long known, lack of robust and accurate tools
RNA modifications example of frequently mentioned potentials with only few applications up to date

Future work based on pipeline backend and baseline set with signal annotation
Base modification framework on event data without dependency to basecaller generation
Base caller based on novel transformer neural networks with impressive results in translation and speech to text







