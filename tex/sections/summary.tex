\chapter{Discussion}
\label{cha:summary}

\textit{But could you not also do that with short reads?} - Expressing an initial scepticism against a novel technology, nanopore sequencing is, six years after the first release in 2014, still termed the emerging technology.
Next-generation sequencing remains the established and widely undisputed reference method.
However, after significant improvements, addressing shortcomings of throughput and accuracy, both third-generation platforms have evolved to become the primary technology for genome assembly and are commonly used for the analysis of viral and bacterial genomes.
Yet, a broader replacement of NGS methods is not in sight, raising the question, why the introductory question is not more frequently re-phrased to: \textit{Would nanopore here not be the more suitable platform?} Or: \textit{Would you not confirm this by using nanopore sequencing?}

The underlying research question of this work is therefore to analyze unique use-cases for nanopore sequencing, identify key limitations and provide a comprehensive investigation of the current status.
Despite many improvements, the application of nanopore sequencing in the context of large mammalian and plant genomes remains challenging.
At the same time are the benefits of long reads known: Reliable alignments in repetitive regions, no duplicates and biases by library amplification and the linkage of distant genetic and epigenetic features on single molecule level.
Yet, the stable throughput, higher read accuracy and established workflows of NGS technologies justify the commitment to nanopore sequencing only for applications infeasible with short reads.

A niche to gain traction at applying nanopore sequencing is the analysis of repetitive regions in the human genome.
For comparison, using short reads, the reachable part of a repeat is limited by the read length, leading to ambiguous alignments, once a read contains only repeat sequence.
Short tandem repeats (STRs) are accumulations of three to six nucleotide long sequence patterns, in disease cases expanded to hundreds of copies.
The epigenetic analysis of repeats is a unique feature of nanopore sequencing, considering that SMRT sequencing is not sufficiently sensitive to 5-methylcytosine on single-molecule level.
Repeat detection by sequencing is the digital advancement over southern blot based quantification, increasing the accuracy and reducing the turnaround time.
\textit{STRique} is a bioinformatic analysis software, developed to integrate the quantification of short tandem repeats with their methylation state on individual read level.
The algorithmic key features of \textit{STRique} are raw signal alignment and the sequence to signal annotation.
Due to the oversampling of the sequencer, raw signal traces are roughly ten times longer compared to their corresponding sequence and, with respect to noise and time warping, generally more difficult to handle.
Yet, only a signal based counting algorithm allows to bypass systematic errors induced during conventional basecalling of tandem repeats.
Hidden Markov Models (HMM) are a powerful resource to align nanopore signals to a reference sequence, with the most prominent usage in the \textit{Nanopolish} package.
However, their computational complexity is scaling quadratically with the number of hidden states, limiting the usage to restricted signal and sequence windows of interest.
The purpose of the signal alignment is therefore, to extract a signal segment covering the repeat with sufficient flanking sequence to anchor a counting HMM.
The flexibility of the HMM is needed over a static alignment, since the repeat length is unknown and varying between reads.
A looping transition around a single repeat profile allows the model to iterate through an arbitrary number of repeat patterns.
The signal alignment is the enabling and at the same time limiting step: The Viterbi path through the HMM is the most likely state-sequence given the observed signal, fed with a wrong signal alignment, the model will still yield a repeat count.
A filtering based on signal alignment scores is therefore crucial to exclude low quality counts.
A useful side-effect of the signal alignment is the ability to mask regions within the read.
\textit{STRique} provides a script to cut out the expanded repeat from the raw signal and creates wild-type like reads passing any conventional downstream pipeline, to detect for instance DNA-methylation in the repeat-flanking sequence.
Taken together, \textit{STRique} is an example for a unique nanopore application, enabling more detailed investigations of short tandem repeat expansions.

Bioinformatics for nanopore sequencing are undergoing constant development.
Even a stable nanopore workflow needs, in contrast to NGS analysis, constant maintenance to factor in accuracy improvements, new pore generations, but also for example recent file format and compression method changes.
Zooming out from individual tool- to pipeline-level, the streamlined processing and data archiving has received very little attention in the nanopore community.
Computational setups ranging from few MinION flow cells analyzed on a Laptop to high-throughput PromethION sequencing in combination with GPU accelerated cluster environments are complicating, if not inhibiting, a uniform processing strategy.
The motivation behind the development of the \textit{Nanopype} pipeline is therefore not only to wrap the processing, but also set up a scaling data management.
While certainly site specific on the implementation detail, the concepts of the pipeline backend transfer to any institute aiming to set up a nanopore workflow.
The ONT device control software \textit{MinKnow} is providing raw data in batches of currently 4k reads packed into \textit{fast5} files.
Further transfer to permanent storage, indexing and archiving are essential for a high-throughput environment.
The organization in batches is, as opposed to a single \textit{fastq} file per NGS run, beneficial for distributed processing.
\textit{Nanopype} operates on batches whenever possible, submitting for instance alignments in batches bears the advantage that multiple compute nodes can process a single sequencing run and that only few reads need to be re-processed in case of errors.
An abstraction layer around the raw data handling is needed to be robust against different file formats (single- and bulk-\textit{fast5}) and changing output directory structures of the \textit{MinKNOW} software.

Using workflow management systems such as \textit{Nextflow} or \textit{Snakemake} over plain shell script implementations greatly simplifies the development and maintenance of a pipeline.
\textit{Snakemake} workflows are becoming more popular for both, standalone pipeline development, but also to enhance the reproducibility of publications by wrapping the applied toolchain into a workflow.
Other pipelines in the field are mostly linear in a way, that they utilize a single tool per processing step and provide specialized outputs to analyze e.g. isoforms or single-nucleotide polymorphisms (SNPs).
Tailored to the requirements of methods development, \textit{Nanopype} provides basic processing with interchangeable methods for each module, supporting for instance \textit{minimap2}, \textit{NGMLR} and \textit{GraphMap} alignments.
Among the greatest advantages of using \textit{Snakemake} is the support of cluster schedulers and the build-in support for \textit{Singularity} containers.
With regard to increasing throughput and fast paced algorithmic accuracy improvements, distributed processing of nanopore datasets is an essential ability to operate on a competitive level.
The deployment of pipeline modules as \textit{Singularity} containers is of less importance for the local usage, but is setting a new state-of-the-art for providing exactly reproducible workflows.
\textit{Nanopype} is unique in not only embedding the order and parameters of applied tools, but also enforcing their version.
Tool installations are commonly left to the user, by either expecting to find required tools in the systems \textit{PATH} variable or by providing options with full path configurations.
Both ways allow the execution of the workflow, but will produce divergent outputs depending on installed versions.
Within \textit{Nanopype}, installation wrappers ensure that tool versions are tied to the pipeline version, guaranteeing consistent outputs across labs and experiments.
Lastly, after building the \textit{Singularity} images as part of the deployment process, functionality tests can be executed with little overhead on minimal example datasets.
The broad functional range of \textit{Nanopype} is ideal for basic processing and serves as a baseline for both, users and developers.
At the same time is the all-in-one approach increasingly complex to maintain and extend.
An example is the trend to apply deep learning algorithms to raw nanopore signals, not only for basecalling, but also for barcode demultiplexing, variant- or base-modification-detection.
Popular neural network frameworks are \textit{Tensorflow} (v1 and v2) and \textit{Pytorch}, which have to be provided by the pipeline.
The current implementation would for instance not allow to have parallel applications requiring different versions of \textit{Tensorflow}.
A subject of future work is therefore the further isolation of included tools, for instance through distinct python virtual environments for compatible applications.
In order to not further increase the complexity of the core pipeline, the \textit{Snakemake} subworkflow feature could be used to implement nested extension pipelines, each requiring \textit{Nanopype} outputs as external inputs.

\textit{Nanopore sequencing: What's coming next?}

The exploration of established use-cases in the literature review reveals the broad applicability of third-generation sequencing for the analysis of bacterial and viral genomes.
Genome assembly and structural variant detection are prominent examples for applications, where NGS sequencing is no longer the primary technology.
The thought experiment of what would be needed to fully replace NGS by nanopore sequencing allows to assess remaining limitations.
If nanopore reads had perfect accuracy, not only on sequence, but also on base-modification level, there would be no further need to store large raw datasets for future re-processing.
A higher accuracy would also reduce the required coverage, facilitating the widespread sequencing of larger genomes with present throughput.
On the other hand are short reads convenient for count-based methods such as ChIP-Seq, where long reads would impede peak-calling.
Viewing third-generation sequencing as an extension, rather than a potential replacement, appears therefore to be the most appropriate perception.

From a hardware and wet-lab perspective, shrinking the sequencing devices using voltage sensors or solid state pores could help to increase throughput and at the same time decrease the amount of input material needed for library preparation.
From a software developer perspective, increasing the sequence and base modfication detection accuracy on existing data would have wide-ranging impact.
Processing infrastructure and signal segmentation methods of this work are the foundation for future work on basecalling and methylation detection algorithms.
Novel transformer neural network architectures have demonstrated impressive results on translation and speech to text conversions.
Translating an event table of feature vectors into a genomic sequence is from an algorithmic point of view very similar and the performance of attention based learning over classical recurrent layers will be exciting to investigate.







