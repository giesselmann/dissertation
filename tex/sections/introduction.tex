\chapter{Introduction}
\label{sec:intro}

%\cleanchapterquote{You can’t do better design with a computer, but you can speed up your work enormously.}{Wim Crouwel}{(Graphic designer and typographer)}


\section{Motivation}
\label{sec:intro:motivation}

\section{Biological Background}
\label{sec:intro:bio}

\begin{itemize}
    \item Organism, cell, DNA
    \item Epigenetics as additional layer
    \begin{itemize}
        \item Briefly histone modifications
        \item DNA methylation, 5mC, hmC, 6mA
    \end{itemize}
    \item 
\end{itemize}


\section{Technical Background}
\label{sec:intro:sequencing}

Third-generation sequencing techniques are currently introducing new perspectives to the field of genome analysis by generating previously unattainable read lengths with averages in the tens of thousands of nucleotides. Among other devices distributed by Oxford Nanopore Technologies (ONT), the MinION in particular is gaining prominence. In brief, the nanopore sequencing process is based on guiding a nucleotide polymer through a pore inserted in a membrane while measuring a change in ionic current as a proxy signal over time. This signal is then interpreted to determine the underlying DNA or RNA sequence. The nanopore technology enables direct readout of sequences from individual DNA or RNA molecules including base modifications since no synthesis or amplification is required.

\begin{itemize}
    \item Sequencing
    \item Breifly generations, 1st, 2nd, 3rd ...
    \item Detail 3rd generation
    \begin{itemize}
        \item Briefly PacBio, Roche
        \item Detail ONT
    \end{itemize}
\end{itemize}

\section{Structure}
\label{sec:intro:structure}


%\textbf{Chapter \ref{sec:nanopype}} \\[0.2em]
%\blindtext

