\pdfbookmark[0]{Abstract}{Abstract}

\vspace*{2mm}
\section*{Abstract}
\label{sec:abstract}

Third-generation long-read technologies denote the latest progression in high throughput DNA and RNA sequence analysis.
Complementing the widespread second-generation short-read platforms, long-read sequencing adds unique application opportunities by generating previously unattainable read lengths.
Despite the remaining higher error rate compared to short reads, single-molecule real-time sequencing (SMRT) and nanopore sequencing advanced to be state-of-the-art for \textit{de-novo} genome assemblies and identification of structural variants.
Continuous throughput and accuracy improvements lead to development of novel methods and applications at a fast pace.
We identify major application fields and key bioinformatic software for long-read sequencing data analysis by employing a data driven literature research.
The integration of citations and keywords into a literature graph provides a scaling approach to analyze an exponentially growing number of third-generation sequencing related publications.
Even though sparking the development of countless bioinformatics software, the streamlined nanopore data processing into standardized formats is still lacking.
As an enabling step for its successful application, we developed \textit{Nanopype}, a modular and scalable pipeline. 
Our approach facilitates the basic steps of basecalling, alignment, methylation- and structural variant detection with exchangeable tools in each module.
Optimized for the usage on high performance compute clusters, we propose a raw data management, capable of handling multiple sequencing devices placed locally and remotely.
Strict version control of integrated tools and deployment as containerized software, ensure reproducibility across projects and laboratories.
Finally, we analyze disease associated repeat regions utilizing targeted nanopore sequencing and the \textit{Nanopype} processing infrastructure.
The expansion of unstable genomic short tandem repeats (STRs) is of particular interest as it causes more than 30 Mendelian human disorders.
Long stretches of repetitive sequence render these regions inaccessible for short-read sequencing by synthesis.
Furthermore, finding current nanopore basecalling algorithms insufficient to resolve the repeat length, we developed \textit{STRique}, a raw nanopore signal based repeat detection and quantification software.
We demonstrate the precise analysis of repeat lengths on patient-derived samples containing C9orf72 and FMR1 repeat expansions. 
The additional integration of repeat- and nearby promoter-methylation levels reveal a repeat length depending gain, suggesting an epigenetic response to the expansion.
Taken together, this work contributes to further increase the usability and provides novel insights based on third-generation nanopore sequencing.




\cleardoublepage
\vspace*{2mm}
\section*{Zusammenfassung}
\label{sec:zusammenfassung}

\begin{otherlanguage}{german}
Im Bereich der DNA und RNA-Sequenzierung stellen Nanopore Technologien den neusten Fortschritt da.
Die Sequenzierung von deutlich längeren Fragmenten eröffnet einzigartige Anwendungsmöglichkeiten im Vergleich zu den weit verbreiteten, synthese-basierten Systemen von zum Beispiel \textit{Illumina}.
Kommerziell verfügbare Plattformen werden zur Zeit von \textit{Pacific Biosciences} (PacBio) und \textit{Oxford Nanopore Technologies} (ONT) vertrieben.
Ungeachtet der höheren Fehlerrate im Vergleich zu bisherigen Systemen hat sich die Nanopore-Sequenzierung zum Stand der Technik für Genom-Assemblierung und zur Identifikation von Strukturvarianten entwickelt.
Das direkte Auslesen chemischer Basen-Modifikationen, insbesondere von 5-Methylcytosin, ermöglicht die Untersuchung von bisher schwer zugänglichen Regionen eines Genoms oder die Verknüpfung von entfernten Merkmalen auf einzelnen Molekülen, was den Einsatz der Nanopore-Sequenzierung in der Epigenetik attraktiv macht.
Eine kontinuierliche Verbesserungen des Durchsatzes und der Genauigkeit führen derzeit zu einer rasanten Entwicklung neuer Methoden und Anwendungen.
Mit Hilfe einer Metadaten basierten Literaturrecherche werden zunächst wichtige Anwendungsfelder und Softwarelösungen für die Analyse von Nanopore Sequenzierdaten identifiziert.
Die Integration von Zitationen und Schlüsselwörtern in einen Literaturgraph bietet einen skalierenden Ansatz, um die exponentiell wachsende Anzahl von Publikationen zu analysieren.
Obwohl die Entwicklung unzähliger Analyseprogramme vorangetrieben wurde, mangelt es immer noch an einer effizienten Verarbeitung von Nanopore-Daten mit standardisierten Dateiformaten.
Als Voraussetzung für eine erfolgreiche Anwendung haben wir daher zunächst \textit{Nanopype} entwickelt, eine modulare und skalierbare Pipeline.
Unser Ansatz ermöglicht es, die grundlegenden Schritte Basecalling, Alignment, Methylierungs- und Variationsdetektierung mit austauschbaren Tools in jedem Schritt durchzuführen.
Optimiert für den Einsatz auf Hochleistungs-Rechenclustern, wird zudem ein Rohdatenmanagement vorgeschlagen, das in der Lage ist, mehrere lokal und entfernt platzierte Sequenziergeräte zu verwalten.
Eine strikte Versionskontrolle der integrierten Tools und die Bereitstellung als Softwarecontainer gewährleisten die Reproduzierbarkeit über Projekte und Labore hinweg.
Schließlich analysieren wir krankheitsassoziierte Variationen genomische Regionen unter Verwendung der Nanopore-Technologie und der Infrastruktur von \textit{Nanopype}.
Die Ausweitung von instabilen kurzen Tandemwiederholungen (Short-Tandem-Repeats, STRs) ist von besonderem Interesse, da sie mehr als 30 menschliche Erkrankungen verursacht.
Lange Abschnitte der repetitiven Sequenz machen diese Regionen unzugänglich für kurze Fragmente aus einer Sequenzierung durch Synthese.
Da die derzeitigen Nanopore-Basecalling-Algorithmen ebenfalls unzureichend sind, um die exakte Wiederholungsanzahl aufzulösen, haben wir \textit{STRique} entwickelt, eine auf dem Nanopore-Rohsignal aufbauende Software zur Erkennung und Quantifizierung von Wiederholungen.
Wir demonstrieren die präzise Bestimmung von Wiederholungslängen an Patientenproben, die C9orf72- und FMR1-Expansionen enthalten.
Ein Zusammenhang zwischen Wiederholungszahl und der erhöhten Methylierung des nahegelegenen C9orf72 Promoter deutet auf eine epigenetische Reaktion auf die Expansion hin.
%Die zusätzliche Verbindung von Wiederholungsanzahl und nahegelegenem Promotor-Methylierungslevel offenbart einen von der Länge abhängigen Zuwachs, was auf eine epigenetische Reaktion auf die Expansion hindeutet.
Zusammengefasst trägt diese Arbeit dazu bei, die generelle Anwendbarkeit der Nanopore-Sequenzierung weiter zu verbessern und demonstriert eine Analyse von repetitiven genomischen Regionen auf Basis des Rohsignals, die in dieser Form mit bisherigen Methoden nicht möglich ist.
\end{otherlanguage}


