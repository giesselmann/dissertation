\pdfbookmark[0]{Abstract}{Abstract}

\chapter*{Abstract}
\label{sec:abstract}
\vspace*{-10mm}

Third-generation long-read technologies denote the latest development in high throughput DNA and RNA sequence analysis.
Complementing the widespread second-generation short-read systems, long-read sequencing add unique application opportunities by generating previously unattainable read lengths.
Commercially available platforms include single-molecule real-time (SMRT) sequencing provided by Pacific Biosciences (PacBio) and nanopore sequencing provided by Oxford Nanopore Technologies (ONT).
Today these are state-of-the-art for de novo genome assemblies and identification of structural variants.
Long RNA reads spanning entire transcripts greatly simplify the isoform detection.

data processing and archiving strategies
sparked the development of novel bioinformatic software

the enabling step for the successful application

methylation in repetitive regions
emerging in epigenetics





\cleardoublepage
\chapter*{Zusammenfassung}
\label{sec:zusammenfassung}
\vspace*{-10mm}


